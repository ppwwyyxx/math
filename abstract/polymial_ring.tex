% $File: polymial_ring.tex
% $Date: Wed Mar 07 10:44:23 2012 +0800
% Author: WuYuxin 
\section{一元多项式环}
K为一数域,$ K[x]$为主理想环,但$ Z[x]$不是

一个$ n$次多项式能被其导数整除$ \Leftrightarrow f(x)=a(cx+b)^n$

必要性:由$ \dfrac{f(x)}{(f(x),f'(x))}=\dfrac{f(x)}{\LM(f'(x))}$
为一次且无重因式即得.
\\

(Sturm)判断一元多项式实根个数:
设$ f_0(x)=f(x), f_1(x)=f'(x)$,做辗转相除:$ f_{s-1}(x)=q_s(x)f_s(x)-f_{s+1}(x)$直至$ f_{s+1}(x)=0$

于是得到序列$ f_0=f,f_1=f',f_2,\cdots,f_s$

记$ V_c$表示序列$ f_i(c),(i=0,\cdots,s)$的变号数,则$ f(x)$在区间$ (a,b),(f(a)f(b)\ne 0)$内的相异实根个数为$ V_a-V_b $

于是可在此区间内求数值解
\\

多项式环上不可约的定义:因式只有可逆元和相伴元.$Q[x] $上不可约也许在$ Z[x]$上可约

$ \alpha $为某个首一整系数多项式的复根,令$ J_{\alpha}=\{ f(x)\in Q[x] | f(\alpha)=0\}$,
$ J_{\alpha}$中次数最低的首一多项式称为$ \alpha $在$ Q$上的极小多项式$ m_{\alpha}(x)$
