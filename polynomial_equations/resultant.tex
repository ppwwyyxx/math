% $File: resultant.tex
% $Date: Sun Jun 03 17:59:27 2012 +0800
% Author: WuYuxin <ppwwyyxxc@gmail.com>
\section{结式方法}
\subsection{原理}
	消元是解方程组的基本原理.
	对于解线性方程组,由于方程只有一次,可以直接将每个方程中的一变元用其余变元表出,再代入消元.
	但对于多项式方程组,由于很多类型的高次方程无根式解,即使有也大多极其复杂,难以使用传统的代入消元手段进行消元,
	而基于换元或方程相加减等的消元方法太过于依赖结构,不具有计算机处理所需的通用性.
	因此需要引入其他的消元方法,结式(Resultant)就是一种.

	首先考虑复数域上关于$ x$的两个多项式方程构成的方程组
	\[  \begin{cases} 
			f(x)=\sum_{i=0}^{n}{a_ix^{n-i}}=0 \\
			g(x)=\sum_{i=0}^{m}{b_ix^{m-i}}=0
	\end{cases} , \texttt{其中} a_0b_0\ne0\]

	这个方程组的解,实际上就是指两多项式的公共复根.

	$ f(x)=0,g(x)=0$有公共复根

	$ \Leftrightarrow \exists f_1(x), g_1(x), \deg f_1(x) < n, \deg g_1(x) < m, \dfrac{f(x)}{f_1(x)} = \dfrac{g(x)}{g_1(x)}$

	$ \Leftrightarrow \exists f_1(x),g_1(x),\deg f_1(x)<n,\deg g_1(x)<m,f(x)g_1(x)=g(x)f_1(x)$

	于是设$ f_1(x)=\sum_{i=0}^{n-1}{u_ix^{n-1-i}},g_1(x)=\sum_{i=0}^{m-1}{v_ix^{m-1-i}}$,
	代入上式,比较左右两边所有$ m+n$个项的系数,可得$ m+n$个线性方程.

	\[ \left \{  \begin{array}{lrclr}
	a_0v_0		&			&= &b_0u_0		&		\\
	a_1v_0+a_0v_1&			&= &b_1u_0+b_0u_1&		\\
	\cdots \cdots &\cdots	&= & \cdots\cdots &\cdots		\\
			&{a_nv_{m-2}+a_{n-1}v_{m-1}}&=&		&b_mu_{n-2}+b_{m-1}u_{n-1}\\
			& a_nv_{m-1}&= &				&b_mu_{n-1}\\
	\end{array}\right . \]

	这个关于$(v_0,\cdots,v_{m-1},u_0,\cdots,u_{n-1})$ 的$ m+n$元齐次线性方程组必须有非零解
	,于是其系数行列式
	
	\[ A=\begin{vmatrix} 
		a_0	&	&	&	&	 &		& b_0 &		&	 &	\\
		a_1	&a_0&	&	&			&	& b_1 & b_0	&	&	\\
  \cdots &\cdots &\cdots &\cdots	&  \cdots&	& \cdots &\cdots	&	\cdots	&	\\
		 &	&	&	&a_n &a_{n-1}   &		&	& b_m  & b_{m-1} \\
		 &	&	&	&	 &a_n		&		&	&	  & b_m\\
	\end{vmatrix} = 0\]

	也即其转置
	\[A^{T}= \begin{vmatrix}
	a_0 & a_1 & \cdots & \cdots & \cdots & \cdots & a_n		&		&	&   \\
		& a_0 & a_1	   & \cdots & \cdots & \cdots & \cdots  &a_n   &	&	\\
		&	  & \cdots & \cdots & \cdots & \cdots & \cdots &		&	&	\\
		&	  &		   & a_0	& a_1	 & \cdots & \cdots & \cdots &	& a_n\\
	b_0 & b_1 & \cdots & b_{m-1}& b_m		 &		  &		   &		&	&   \\
		& b_0 & b_1	   & \cdots  &\cdots &b_m   	&		&		&	&	\\
		&	  & \cdots & \cdots & \cdots & \cdots & \cdots &		&	&	\\
		&	  &		   &		&		&   b_0   & b_1		& b_2	& \cdots & b_m\\
	\end{vmatrix} =0\]
	注意到$ A^{T}$仅由多项式$ f,g$的系数决定,可记$ Res_x(f,g)=|A^{T}|,$
	称其为多项式$ f,g$的{\bf Sylvester结式(Sylvester's Resultant) },简称结式.

	根据上述讨论,我们有结论:$ f(x),g(x)$有重根$ \Leftrightarrow Res_x(f,g) = 0$
	
	\vspace{13pt}
	再考虑二元多项式方程组:

	$ \begin{cases}
		f(x,y)=0 \\g(x,y)=0 
	\end{cases}$,以$ x$为主元,视$ y$为常数,可求得结式$ P(y) = R_x(f,g)$,是$ y$的多项式.

	由之前对一元方程组的讨论可知,对$ P(y)$的每个零点$ y_0,f(x,y_0),g(x,y_0)$有公共复根.
	反之,对于原方程组的每个解$ (x_0,y_0), $必有$ P(y_0) = 0$

	于是,对原方程组的求解转化为对一元方程$ P(y) = 0$的求解,成功实现了消元,降低了问题的难度.
	
	\vspace{13pt}
	对两个方程构成的含有多个变元的方程组,可以使用相同手段消去其中某个变元.
	而对于超过两个方程构成的方程组,可以每次取其中两个方程计算结式,例如解如下方程组:
	\[  \begin{cases} f_1 = xy+z-5 = 0\\ f_2 = x+y+z-6 = 0 \\ f_3 = x^2-2xy+y^2-2z = 0
	\end{cases}\]

	$ Res_x(f_1,f_2) = y^2+(z-6)y-z+5 $ 

	$ Res_x(f_2,f_3)=4y^2+(4z-24)y + z^2-14z+36$

	$ Res_y(Res_x(f_1,f_2),Res_x(f_2,f_3)) = (z-2)^2(z-8)^2$

	于是消去了$ y,z$,得到$ z=2,z=8,$代回原方程可求得完整解.

	对于多个方程构成的方程组,也有多元结式理论可以用于解决消元问题.\cite{ideals}

\subsection{应用}
	对于一元方程组,结式方法的一个重要应用是判断多项式是否有重根.
	考虑多项式$ f(x) = \prod_{i=1}^{k}{(x-x_i)^{r_i}}$,
	显然对于每个$ i$,有 $ r_i >1 \Leftrightarrow  f'(x_i) = 0 \Leftrightarrow x_i $ 是$f(x)$与$ f'(x)$的公共复根.
	所以, $ f(x)=0$有重根$ \Leftrightarrow Res(f,f')=0$.
	
	\vspace{5pt}
	另外,上述的二元方程组的结式求解方法可应用于物理解题中常常能遇到的一类代数曲面/曲线隐式化问题.
	例如,用参数方程表示的曲线
	$ \begin{cases}x=\dfrac{-t^2+2t}{t^2+1} \\ y=\dfrac{2t^2+2t}{t^2+1} \end{cases}$

	将两式都写为关于$ t$的多项式: 
	$\begin{cases}f(t)=(x+1)t^2 -2t +x\\ g(t)=(y-2)t^2-2t+y \end{cases}$

	\vspace{5pt}
	排除掉$ x+1=y-2=0$的情形,计算结式$ Res_t(f,g)= 8x^2-4xy+5y^2+12x-12y$ 
	
	于是$8x^2-4xy+5y^2+12x-12y=0,(x,y) \ne (-1,2)$为曲线隐式化后的结果.
