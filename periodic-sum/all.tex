%File: all.tex
%Date: Tue Oct 15 15:40:56 2013 +0800
%Author: Yuxin Wu 

设$ x_1(t), x_2(t)$为$\mathbb{R}$上的\textbf{连续}周期函数, 最小正周期分别为$ T_1, T_2$.
$x(t) = x_1(t) + x_2(t)$, 则:
$x(t) $为周期函数的\textbf{充要条件}是$\dfrac{T_1}{T_2} \in \mathbb{Q}$


\begin{proof}
  充分性显然, 下证必要性.
  设$x(t)$为周期函数, $T$是它的一个正周期, 则:
  \[ x(t+T) \equiv x(t) \Leftrightarrow x_1(t+T) - x_1(t) \equiv x_2(t) - x_2(t+T)\]

设$ f(t) = x_1(t+T) - x_1(t) = x_2(t) - x_2(t+T) $,
显然$ f(t)$也为连续周期函数, $ T_1, T_2$均是$ f(t)$的周期.
(此时还无法得出$ \dfrac{T_1}{T_2}\in \mathbb{Q}$, 因为未证明$ f(t)$存在最小正周期.)

接下来分情况讨论:
\begin{enumerate}
  \item $f(t) \equiv 0 $.

    此时易见$T $是$ x_1(t), x_2(t)$的公共周期, 而已知$x_1(t), x_2(t) $有最小正周期$T_1, T_2 $, 因而必须有:
    \[ \dfrac{T}{T_1} \in \mathbb{Z}, \dfrac{T}{T_2} \in \mathbb{Z} \Rightarrow \dfrac{T_1}{T_2} \in \mathbb{Q}\]
  得证.

\item $f(t) \equiv C \neq 0$
  \begin{align*}
    \left \{
      \begin{array}{lll}
    x_1(t+T) - x_1(t) &=& C \\
    x_1(t+2T) - x_1(t + T) &=& C \\
    &\vdots \\
    x_1(t+nT) - x_1(t +(n-1) T) &=& C
      \end{array}
    \right. \Rightarrow  x_1(t+nT) = x_1(t) + (n-1)C
  \end{align*}
  \[ \Rightarrow \lim\limits_{n\to+\infty}x_1(t+nT) = \infty, \forall t \in \mathbb{R} \]

无界性与$x_1(t) $的连续性,周期性矛盾, 故不可能.

\item $f(t) $ 非常函数.

  此时若能证明$f(t)$存在最小正周期$T_0$, 则必有
  \[ \dfrac{T_1}{T_0} \in \mathbb{Z}, \dfrac{T_2}{T_0} \in \mathbb{Z} \Rightarrow \dfrac{T_1}{T_2} \in \mathbb{Q}.\]

前已证明$ f(t)$是连续周期函数, 设其所有正周期的集合为$ S,\  T' = \inf S > 0$, 则只需证明$ T' \in S$.

由$ T' = \inf S$, 故$S$中存在点列$s_0, s_1,\cdots $收敛于$ T'$, 且由$S$ 的定义,
\[  f(t+s_i) = f(t), \forall i \in \mathbb{N}, t \in \mathbb{R}.  \]
对上式取极限$n\to +\infty$, 由$f$的连续性, 立刻得$ f(t+T') = f(t)$, 故$ T' \in S$, 是$f$的最小正周期.
\end{enumerate}

\end{proof}

\vspace{2em}

$ \dfrac{T_1}{T_2} \in \mathbb{Q}$满足时, $ x(t)$的最小正周期\textbf{不一定}是$ T_1, T_2$的最小公倍数$ [T_1, T_2]$.
\begin{proof}
首先$ x(t)$未必有最小正周期..

只考虑$ x(t)$存在最小正周期的情况, 若上结论成立, 考虑下式:

\[ x(t) + (-x_1(t)) = x_2(t)\]

则$ x_2(t)$的最小正周期应当是$[[T_1, T_2], T_1] = [T_1, T_2]$, 而不是$T_2$. 矛盾.

反例也很容易由此构造.
\end{proof}
