% $File: wilson.tex
% $Date: Wed Mar 07 10:42:48 2012 +0800
% Author:  ppwwyyxxc@gmail.com

\section{Wilson}
Wilson定理:素数$ p \Leftrightarrow (p-1)!\equiv -1 \pmod p$

可推出:$ (p-k)!(k-1)!\equiv (-1)^k \pmod p$

Lagrange定理:$ f(x)=\sum_{i=1}^{n}{a_ix^i},p \nmid a_i$,则$ n$次同余方程
$ f(x) \equiv 0  \pmod p$的解数$  \le n$

对$ n$归纳反证.假设$ n+1$个解$ c_1\cdots c_{n+1}$,则$ f(x)-f(c_1)=(x-c_1)h(x)$

于是$ c_2,\cdots c_{n+1}$均为$ n-1$次同余方程$ h(x)\equiv 0 \pmod p$的解.矛盾

{\bf 推论 }:若$ f(x)\equiv 0$的解数$ >n$,则各项系数均被$ p$整除.
\\

$ f(x)=(x-1)(x-2)\cdots (x-p+1)=\sum_{i=0}^{p-1}{s_ix^i}\equiv x^{p-1}-1 \pmod p $(Fermat)

$ \Rightarrow f(x)-x^{p-1}+1=\sum_{i=1}^{p-2}{s_ix^i}+(p-1)!+1\equiv 0 \pmod p$ 

由Lagrange得$ p\mid s_i,1\le i \le p-2$
\\

$ f(x)=f(p-x)\Rightarrow f(-x)=f(p+x)$ 

$ \Rightarrow x^{p-1}+\sum_{i=1}^{p-2}{(-1)^is_ix^i}=(p+x)^{p-1}+\sum_{i=1}^{p-2}{s_i(p+x)^i}$

两边模$ p^2$得,$ x^{p-1}+\sum_{i=1}^{p-2}{(-1)^is_ix^i}\equiv x^{p-1}+(p-1)px^{p-2}+\sum_{i=1}^{p-2}{s_ix^i}$

$ \Rightarrow \sum_{i=1}^{p-2}{[(-1)^i-1]s_ix^i}\equiv p(p-1)x^{p-2}\pmod {p^2}$ 

$ \Rightarrow \sum_{i=1}^{p-3}{[(-1)^i-1]s_ix^i}\equiv 0 \pmod {p^2}(\because s_{p-2}=-\dfrac{p(p-1)}{2})$

$ \Rightarrow p^2 \mid s_1,s_3,\cdots s_{p-4}$

{\bf 推论 }:$ p^2 \mid s_1=(p-1)!(1+\dfrac{1}{2}+\cdots+\dfrac{1}{p-1}),p \mid s_{p-3}=\sum_{1\le i\le j\le p-1}{ij}$
\\

Wilson定理推广:

T1.奇素数$ p$,设$ c=\varphi(p^l),r_1,\cdots,r_c$是mod$ p^l$的缩系,则$ \prod_{i=1}^{c}{r_i}\equiv -1 \pmod{p^l}$

{\bf 证 }:对每个$ r_i$有唯一$ r_j$使$ r_ir_j\equiv1\pmod{p^l}.$

此时$ r_i=r_j\Leftrightarrow r_i\equiv 1,-1\pmod{p^l}$配对即得证.
\\

T2:$ \because \varphi(p^l)=\varphi(2p^l),$取$ r_i'= \begin{cases}r_i,& 2 \nmid r_i\\ r_i+p^l,& 2\mid r_i\end{cases} $,则$ r_i'$为mod$ 2p^l$的缩系

且$ \prod_{i=1}^{c}{r_i'}\equiv -1 \pmod{p^l},2 \mid  \prod_{i=1}^{c}{r_i'}+1\Rightarrow  \prod_{i=1}^{c}{r_i'}\equiv -1 \pmod{2p^l}\\$
\\

T3:设$ c=\varphi(2^l),l\ge 3,r_1\cdots r_c$是mod$ 2^l$的缩系.则$  \prod_{i=1}^{c}{r_i}\equiv 1 \pmod{2^l}$

{\bf 证 }:同T1,使$ r_i=r_j$的充要条件是$ \dfrac{r_i-1}{2}\dfrac{r_i+1}{2}\equiv 0 \pmod{2^{l-2}}\Leftrightarrow r_i\equiv 1,2^{l-1}\pm 1,2^l-1$
