% $File: number-theory.tex
% $Date: Fri Nov 09 11:27:20 2012 +0800
% $Author: wyx 

\documentclass[a4paper]{article}
\usepackage{fontspec,zhspacing,amsmath,amssymb,verbatim,minted,zhmath}
\usepackage[hyperfootnotes=false,colorlinks,linkcolor=blue,anchorcolor=blue,citecolor=blue]{hyperref}
\usepackage[sorting=none]{biblatex}
%\usepackage[dvips]{graphicx}
\usepackage{subfigure}
\usepackage{indentfirst}
\zhspacing
\setlength{\parindent}{1em}
\renewcommand{\baselinestretch}{1.4}

\renewcommand{\abstractname}{摘要}
\renewcommand{\contentsname}{目录}
\renewcommand{\figurename}{图}
\defbibheading{bibliography}{\section{参考文献}}
\bibliography{refs.bib}
% \figref{label}: reference to a figure
\newcommand{\figref}[1]{\hyperref[fig:#1]{图\ref*{fig:#1}}}
% \secref{label}: reference to a section
\newcommand{\secref}[1]{\hyperref[sec:#1]{\ref*{sec:#1}节}}

\newtheorem{theorem}{Theorem}[section]
\newtheorem{lemma}[theorem]{Lemma}
\newtheorem{proposition}[theorem]{Proposition}
\newtheorem{corollary}[theorem]{Corollary}

\let\Oldsum\sum
\renewcommand{\sum}{\displaystyle\Oldsum}
\let\Oldprod\prod
\renewcommand{\prod}{\displaystyle\Oldprod}
\newcommand{\qed}{\hfill \ensuremath{\Box}}

\newenvironment{proof}[1][Proof]{\begin{trivlist}
\item[\hskip \labelsep {\bfseries #1}]}{\end{trivlist}}
\newenvironment{definition}[1][Definition]{\begin{trivlist}
\item[\hskip \labelsep {\bfseries #1}]}{\end{trivlist}}
\newenvironment{example}[1][Example]{\begin{trivlist}
\item[\hskip \labelsep {\bfseries #1}]}{\end{trivlist}}
\newenvironment{remark}[1][Remark]{\begin{trivlist}
\item[\hskip \labelsep {\bfseries #1}]}{\end{trivlist}}

%\newcommand{\qed}{\nobreak \ifvmode \relax \else
%      \ifdim\lastskip<1.5em \hskip-\lastskip
%      \hskip1.5em plus0em minus0.5em \fi \nobreak
%      \vrule height0.75em width0.5em depth0.25em\fi}
%




%% $File: mint-defs.tex
% $Date: Fri Mar 22 21:26:39 2013 +0800
% $Author: wyx 


% \inputmintedConfigured[additional minted options]{lang}{file path}{
\newcommand{\inputmintedConfigured}[3][]{\inputminted[fontsize=\footnotesize,
	label=#3,linenos,frame=lines,framesep=0.8em,tabsize=4,#1]{#2}{#3}}

% \phpsrc[additional minted options]{file path}: show highlighted php source
\newcommand{\phpsrc}[2][]{\inputmintedConfigured[#1]{php}{#2}}
% \phpsrcpart[additional minted options]{file path}{first line}{last line}: show part of highlighted php source
\newcommand{\phpsrcpart}[4][]{\phpsrc[firstline=#3,firstnumber=#3,lastline=#4,#1]{#2}}
% \phpsrceg{example id}
\newcommand{\phpeg}[1]{\inputminted[startinline,
	firstline=2,lastline=2]{php}{res/php-src-eg/#1.php}}

\newcommand{\txtsrc}[2][]{\inputmintedConfigured[#1]{text}{#2}}
\newcommand{\txtsrcpart}[4][]{\txtsrc[firstline=#3,firstnumber=#3,lastline=#4,#1]{#2}}

\newcommand{\pysrc}[2][]{\inputmintedConfigured[#1]{py}{#2}}
\newcommand{\pysrcpart}[4][]{\pysrc[firstline=#3,firstnumber=#3,lastline=#4,#1]{#2}}

\newcommand{\confsrc}[2][]{\inputmintedConfigured[#1]{squidconf}{#2}}
\newcommand{\confsrcpart}[4][]{\confsrc[firstline=#3,firstnumber=#3,lastline=#4,#1]{#2}}

\newcommand{\cppsrc}[2][]{\inputmintedConfigured[#1]{cpp}{#2}}
\newcommand{\cppsrcpart}[4][]{\cppsrc[firstline=#3,firstnumber=#3,lastline=#4,#1]{#2}}

\renewcommand{\P}[1]{\text{P}\left(#1\right)}
\renewcommand{\Pr}[1]{\text{P}\left\{#1\right\}}
\newcommand{\Px}[2]{\text{P}_{#1}\left(#2\right)}
\newcommand{\E}[1]{\text{E}\left[#1\right]}
\newcommand{\Var}[1]{\text{Var}\left[#1\right]}
\renewcommand{\T}[1]{\Theta\left(#1\right)}
\renewcommand{\O}[1]{\text{O}\left(#1\right)}
\renewcommand{\d}[1]{\text{d}\,#1}
\newcommand{\real}{\mathbb{R}}
\newcommand{\card}[1]{\left\|#1\right\|}
\newtheorem{lemma}{Lemma}
\newtheorem{theorem}{Theorem}

\newcommand{\Stir}[2]{\left\{\begin{matrix}#1\\#2\end{matrix}\right\}}


\title{Number Theory}
\author{ppwwyyxxc@gmail.com}

\begin{document}
\maketitle

\tableofcontents
% $File: order.tex
% $Date: Wed Mar 07 10:39:46 2012 +0800
% $Author: wyx  

\section{Order}
Definition: $  \delta_m(a)=\min \{x|a^x \equiv 1 \pmod m\} $ 

推广: $ a^d\equiv b^d \pmod p$,取倒数$ bb'\equiv1\pmod p$,则$ d=\delta_p(ab')$.性质类似
\\
若$ a^n \equiv 1 \pmod m$ ,则 $ \delta_m(a)\mid n $ .
否则设$ n=\delta_m(a)q+r,a^r \equiv a^n \equiv 1 $且$ r<\delta_m(a)$.矛盾

特别地,若 $ a^p\equiv 1 \pmod m$ , 则 $ \delta_m(a)=1 $ 或$ p$ 

Mersenne's Prime的因子特征:$ q\mid 2^p-1\Rightarrow p=\delta_q(2)\mid (q-1)\Rightarrow q\equiv 1 \pmod{2p} $ 
\\

$ (a,p)=1$,则在$ p^0,p^1,\ldots p^{a-1} \pmod a$中抽屉得 $ \exists d \le a-1: a|p^d-1\Rightarrow \delta_p(a)\le a-1$
\\

证明$ n \nmid 2^n-1$:

设$ n$最小素因子$ p$,则$ \delta_p(2) \mid (p-1,n)=(p-1,\dfrac{n}{p^{\alpha}})=1$.

或者利用递降:$ n\rightarrow \delta_n(2) ; (a,b)\rightarrow (b,(a,b))$
\\

$ n\mid 2^n+1\Rightarrow \delta_p(2)\mid (2n,p-1)=(2,p-1)\Rightarrow p=3$.

事实上有$ 3^k \mid 2^{3^k}+1$,以及$ n\mid 2^n+1 \Rightarrow m\mid 2^m+1,m=2^n+1$
\\

反证$ n \nmid m^{n-1}+1$:

设$ n-1=2^kt\Rightarrow m^{2^kt}\equiv -1 \pmod p\Rightarrow \delta_p(m)\nmid 2^kt,\delta_p(m)\mid 2^{k+1}t\Rightarrow 2^{k+1}\mid\delta_p(m)$

又$ \delta_p(m)\mid p-1,\therefore p \equiv 1 \pmod{2^{k+1}}.$
考虑到$ p$为$ n$任意素因子$ \Rightarrow n\equiv 1 \pmod{2^{k+1}}$,与$ n-1=2^kt$矛盾
\\
\\

关于$ r_k=\delta_{p^k}(a)$的求解($ p$为奇数).设$ p^{k_0}\parallel a^{r_1}-1$

i)当$ 1\le k \le k_0$时,$ a^{r_k}\equiv1 \pmod{p^k \rightarrow p}\Rightarrow r_1\mid r_k$

$ a^r\equiv1 \pmod{p^{k_0}\rightarrow p^k}\Rightarrow r_k\mid r_1. \therefore r_k=r_1$

ii)当$ k\ge k_0$时,对$k $归纳证明$ r_k=r_1 p^{k-k_0}$

引理:$ p^{k_0+i}\parallel a^{r_1p^i}-1\Leftrightarrow a^{r_1p^i}=1+p^{k_0+i}u,(u,p)=1$.

{\bf 证明 }:归纳.
$ a^{r_1p^{i+1}}=(a^{r_1p^i})^p=(1+p^{k_0+i}u)^p=1+p^{k_0+i+1}(1+C_p^2u^2p^{k_0+i-1})$

引理中取$ i=k-k_0,$则$ a^{r_1p^{k-k_0}}\equiv 1 \pmod{p^k}\Rightarrow r_k \mid r_1p^{k-k_0}$

$ a^{r_k}\equiv 1\pmod{p^k \rightarrow p^{k-1}}\Rightarrow r_{k-1}\mid r_k \therefore r_1p^{k-k_0-1}\mid r_k \mid r_1p^{k-k_0}$

再取$ i=k-k_0-1$,由$ p^{k-1}\parallel a^{r_1p^{k-k_0-1}}-1$知$ a^{r_1p^{k-k_0-1}}\not \equiv 1 \pmod{p^k}.$


$  \therefore r_k=
\begin{cases}
r_1, & 1\le k \le k_0 \\
r_1p^{k-k_0}, & k\ge k_0 
\end{cases} 
$
\\

$ r_k=\delta_{2^k}(a)$的求解:

i)$ a=4k+1,2^{k_0}\parallel a-1,r_k=
\begin{cases}
1,& 1\le k \le k_0 \\ 
2^{k-k_0},& k\ge k_0 
\end{cases}
$ 

ii)$ a=4k+3,2^{k_0}\parallel a+1,r_k=
\begin{cases} 
1,& k=1 \\ 
2,& 2\le k\le k_0+1 \\ 
2^{k-k_0},& k\ge k_0+1 
\end{cases}$
\\

引理的推广:$ a^{mrp^i}=1+p^{k_0+i}u,(u,p)=1.$

设$ n=mrp^i$可得一命题:$ r=\delta_p(a),r\mid n,p^{\alpha}\parallel n\Rightarrow p^{\alpha}\parallel \dfrac{a^n-1}{a^r-1} $
\\

{\bf 反证 }:对给定$ n,a$,不存在无穷个$ k,s.t.n^k\mid a^k-1$

i)$ n$含奇因子$ p,a^k\equiv 1 \pmod {p^k}\Rightarrow r_k=r_1p^{k-k_0}\mid k\Rightarrow k>r_1p^{k-k_0}\ge 3^{k-k_0}$不可能无穷个

ii)若$ k$为奇,则$ 2^k\mid a^k-1\Rightarrow 2^k\mid a-1,$只有有限个$ k$.

若$ k$为偶,$ a^{2l}\equiv 1 \pmod{2^l}.$当$ l>k_0$时,$ 2^{l-k_0}\mid l$不可能无穷个.
\\
\\
$ r_k=\delta_m(a^k)=\dfrac{r_1}{(r_1,k)}$.

{\bf 证 }:设$ r'=\dfrac{r_1}{(r_1,k)}$.显然$ (r',\dfrac{k}{(r_1,k)})=1$

由定义,$ a^{kr_k}\equiv 1 \pmod m, a^{kr'}\equiv 1 \pmod m. \Rightarrow  r_1 \mid kr_k,r_k\mid r'$

$ \therefore r' =\dfrac{r_1}{(r_1,k)}\mid \dfrac{k}{(r_1,k)}r_k\Rightarrow r' \mid r_k. \therefore r'=r_k$

{\bf 推论 }:有$ \varphi(r_1)$个$ k,s.t.(r_1,k)=1.$又$ a^0,a^1,\cdots,a^{r_1-1}$对模$ m$不同余

所以其中至少有$ \varphi(r_1)$个$ k,s.t.\delta_m(a^k)=r_1$.

即在模$ m$的一个缩系中至少有$ \varphi(r_1)$个$ k,s.t.r_k=r_1$
\\

若$ (m_1,m_2)=1,$则$ \delta_{m_1m_2}(a)=[\delta_{m_1}(a),\delta_{m_2}(a)]=[r_1,r_2]$

{\bf 证 }:i)显然对$ \forall n \mid m,\delta_n(a)\mid \delta_m(a). \therefore [r_1,r_2] \mid \delta_{m_1m_2}(a)$ 

ii)$ a^{[r_1,r_2]}\equiv 1 \pmod{m_1,m_2 \rightarrow m_1m_2}\Rightarrow \delta_{m_1m_2}\mid [r_1,r_2]$

{\bf 推论 }:$ (m_1,m_2)=1$,则对$ \forall a_1,a_2,\exists a,s.t.\delta_{m_1m_2}(a)=[\delta_{m_1}(a_1),\delta_{m_2}(a_2)]$

{\bf 证 }:取$ a\equiv a_i \pmod{m_i},i=1,2$.则$ \delta_{m_i}(a)=\delta_{m_i}(a_i)$.由原命题即证.
\\

$ \min\{ n|2^n\equiv -1 \pmod p\}<\delta_p(2),$否则 $,2^{n-\delta_p(2)}\equiv 2^n \equiv -1$,与最小性矛盾.
\\

$ p=3k+2$时,$ x$取$ \mod p$完系,则$ x^3$亦遍历.否则$ x^3\equiv y^3\Rightarrow \delta_p(xy^{-1})\mid(3,p-1)=1$.矛盾
\\

无穷数列$  \dfrac{1}{9}(10^{k\delta_{9a}(10)}-1)(k\ge1) $ 中,每项均由1组成且均为$ a$的倍数
\\

奇素$ p,p^n|a^p-1\Rightarrow p^{n-1}|a-1$
\\

$ \exists n,s.t.p\parallel2^n-1\Rightarrow p\parallel 2^{p-1}-1$

证:假设$ p^2 \mid 2^{p-1}-1\Rightarrow \delta_{p^2}(2)\mid p-1.$

又$ 2^{pn}-1=(2^n-1)(2^{n(p-1)}+2^{n(p-2)}+\cdots+2^n+1)\equiv(2^n-1)p\equiv 0\pmod{p^2}$

$ \therefore \delta_{p^2}(2)\mid (pn,p-1)=(n,p-1)\mid n\Rightarrow 2^n\equiv 1 \pmod {p^2}$.矛盾
\\

奇素数$ p,{pn+1}$中含无穷多素数:

证:取$ x^p-1$的因子$ q,s.t.q \nmid x-1$ (why can?).则$ \delta_q(x)=p$.

设$ (q-1,p)=d$ ,则$ \exists u,v,s.t.u(q-1)+vp=d\Rightarrow x^d\equiv(x^{q-1})^n(x^p)^v\equiv 1 \pmod q\Rightarrow d=p$ 

$ \therefore p\mid q-1\Leftrightarrow q=pn+1$.又$ \dfrac{x^p-1}{x-1}$含无穷个素因子$ q$,可知$ {pn+1}$中有无穷多素数

\input{wilson.tex}
\input{special_numbers.tex}
\input{function.tex}
\input{gauss.tex}
\input{diophantine.tex}
% $File: polynomial.tex
% $Date: Wed Mar 07 14:30:49 2012 +0800
% Author: WuYuxin 
\section{多项式}
$ f(x)$次数$ \le n$,且$ f(k)(k=0\cdots n)$均为整数,

则$ f(x)$为整值多项式,且整值多项式必可表为$ \sum_{i=0}^{n}{a_iC_x^i}$

{\bf 证 }:设$ f(x)=\sum_{i=0}^{n}{a_iC_x^i},a_i \in \mathbb{C}$,取$ x=0,1,\cdots$

$ \mathbb{Z}\ni f(0)=a_0$.又$ \mathbb{Z}\ni f(1)=a_0+a_1\Rightarrow a_1\in \mathbb{Z}\cdots a_i\in \mathbb{Z}$

{\bf 推论 }:$ f(x)$次数$ \le n$,且对连续$ n+1$个整自变量取整值,则其为整值多项式(平移即可)
\\

整系数多项式$ P(x):u-v \mid P(u)-P(v)\Rightarrow P(1)\equiv P(k+1) \equiv \cdots \equiv P(nk+1) \pmod k$
\\

设有$ a_1,\cdots a_m$满足对$ \forall n,\exists i,a_i\mid F(n),$则$ \exists i,\forall n,a_i\mid F(n)$

{\bf 反证 }:设$ \exists x_1, a_1 \nmid F(x_1),\cdots \exists x_m, a_m \nmid F(x_m)
\Leftrightarrow \exists d_i=p_i^{r_i},d_i \mid a_i$且$ d_i\nmid F(x_i)$

$ d_1,\cdots d_m$中同底数只保留低次幂,得$ d_1\cdots d_s.$则$ \exists N ,\forall i,N\equiv x_i \pmod{d_i}$

$ \therefore \forall i,F(N)\equiv F(x_i)\not\equiv 0 \pmod{d_i}\Rightarrow \forall i,F(N)\not\equiv 0 \pmod{a_i}$
\\

设素数$ p_1,\cdots p_k,\forall i ,\exists x_i,p_i \mid P(x_i)\Rightarrow \exists x ,\prod_{i=1}^{k}{p_i}\mid P(x)$

{\bf 证 }:孙子.$ x\equiv x_i \pmod)p_i\Rightarrow P(x)\equiv P(x_i)\equiv 0$
\\

整系数多项式$ P(x)=a_nx^n+\cdots a_1x\pm1$值域的素因子无穷:假设有限-$ p_1\cdots p_k$

则$ P(i\prod{p_t})$不含素因子$ \Rightarrow P(i\prod{p_t})=\pm 1$但$ n$次多项式至多给出$ 2n$个$ \pm 1$
\\

(Gauss)本原多项式的乘积仍是本原多项式.

{\bf 证 }:设$ f(x)g(x)$各项系数$ c_k$有公因子$ p$,设$ i=\min\{t: p \nmid a_t\},j=\min\{t: p\nmid b_t\}$

则由$ c_{i+j}$的展开可得矛盾

进一步,记各项系数的gcd(多项式的容度)为$ c(f)$,有$ c(fg)=c(f)c(g)$
\\

(Eisenstein)$ f(x)=\sum_{i=0}^{n}{a_nx^n},\exists p\in P,p \nmid a_n,p^2\nmid a_0,p \mid a_0\cdots a_{n-1}\Rightarrow f$不可约

{\bf 证 }:设$ f(x)=\sum_{i=0}^{s}{b_ix^i}\sum_{i=0}^{t}{c_ix^i},$不妨设$ p \mid b_0,$显然有$ p\nmid c_0,p\nmid b_n$

设$ i=\min\{ t:p\nmid b_t\}$,考虑$ a_i$的展开可得矛盾

证$ p$ 阶分圆多项式不可约:取$ x=y+1$

% $File: n=ax+by.tex
% $Date: Thu Mar 08 20:51:30 2012 +0800
% Author: WuYuxin 
\section{表n为ax+by}
$(a,b)=1\Rightarrow \exists x,y\in \mathbb{N}^+,ax-by=1$
\\

$ \forall n>ab-a-b$可表为$ ax+by,x,y\in \mathbb{N}$.

{\bf 证 }:设$ n=a(x_0+bt)+b(y_0-at),$可取$ t$使得$ 0 \le y =y_0-at\le a-1$

则$ ax=n-(y_0-at)b>ab-a-b-(a-1)b=-a\Rightarrow x>-1\Rightarrow x\in \mathbb{N}$
\\

$ n=ab-a-b$不可表.反设结论不成立,则
$ ab=(x+1)a+(y+1)b\Rightarrow b\mid x+1\Rightarrow x+1\ge b$矛盾
\\

写$ n$为$ ax+by,0\le x\le b-1$,若$ n=ax+by$中$ y\ge 0$

则 $ n'=(b-1-x)a+(-1-y)b$中仍有$ 0 \le b-1-x \le b-1$,但$ -1-y <0$.

于是$ n$可表$ \Rightarrow ab-a-b-n$不可表.$ \therefore [0,ab-a-b]$
中有$ \dfrac{(a-1)(b-1)}{2}$个不可表.
\\

在矩形$\begin{matrix} 0\le x\le b\\ 0\le y\le a \end{matrix} $中有$ (a+1)(b+1)$个整点.

其中使$ 0\le ax+by<ab$的整点有$ \dfrac{(a+1)(b+1)}{2} -1$个
\\

$ n=ax+by,x,y\in \mathbb{N^+}$有至少两种表法
$ \Leftrightarrow n$可表为$ ab+a+b+ax+by,x,y\in \mathbb{N}$

i) $ ab+a+b+ax+by=a(1+b+x)+b(1+y)=b(1+a+y)+a(1+x)$

ii) $ ax_1+by_1=ax_2+by_2\Rightarrow a\mid y_2-y_1\Rightarrow y_2\ge a+1$

$ \therefore ax_2+by_2=ab+a+b+(y_2-a-1)b+(x_2-1)a$

% $File: quadratic_residue.tex
% $Date: Thu Aug 02 18:26:01 2012 +0800
% Author: WuYuxin 
\section{Quadratic Residue}
Def: $ \exists x,x^2\equiv d \pmod p,d<p,p$为奇素数.

T1: $ \pmod p$ 的一个缩系中有$	\dfrac{p-1}{2}$个$ \pmod p$的二次剩余与二次非剩余,
且方程$ x^2 \equiv d \pmod p$若有解必有两解.

{\bf 证}:取绝对最小缩系$ S = \{ -\dfrac{p-1}{2},\cdots -1, 1,\cdots \dfrac{p-1}{2}\}.$ 

$ (\dfrac{d}{p})= 1\Leftrightarrow d \equiv 1^2,\cdots,(\dfrac{p-1}{2})^2 \pmod p. $于是有$ \dfrac{p-1}{2}$个二次剩余
\\

T2(Euler):$ ( \dfrac{d}{p} )\equiv d^{\frac{p-1}{2}} \pmod p$.
(由Fermat定理显然$ d^{\frac{p-1}{2}}\equiv \pm 1 \pmod p$)

{\bf 证:}i)若$ (\dfrac{d}{p})=1,$则$ \exists x_0^2\equiv d\Rightarrow d^{\frac{p-1}{2}}\equiv x_0^{p-1}\equiv 1$

ii)对$ p \nmid d$ ,满足$ ax\equiv d\pmod p$的缩系中的$ a,x$一一对应.

假设$ (\dfrac{d}{p})=-1,$则总有$ a\ne x$,则$ d^{\frac{p-1}{2}}\equiv \prod_{i=1}^{\frac{p-1}{2}}{(a_ix_i)}\equiv (p-1)!\equiv -1\pmod p$
\\

T3(Gauss引理):设对$ 1\le j < \dfrac{p}{2},t_j\equiv jd \pmod p$且$ 0<t_j<p$.
设在$ t_1\ldots,t_{\frac{p-1}{2}}$中有$ n$个$ >\dfrac{p}{2}$,则$ (\dfrac{d}{p})=(-1)^n$

{\bf 证}:设$ >\dfrac{p}{2}$的为$ r_1,\ldots,r_n,<\dfrac{p}{2}$的为$ s_1,\ldots,s_k.k+n=\dfrac{p-1}{2}$.

由于$ \forall 1\le j<i<\dfrac{p}{2},t_j\pm t_i\equiv (j\pm i)d \not \equiv 0\Rightarrow t_j\not\equiv \pm t_i\Rightarrow s_j\not\equiv -r_i\pmod p$

又$ 1\le p-r_i<\dfrac{p}{2},$于是$ s_1,\ldots,s_k,p-r_1,\ldots,p-r_n$为$ 1,2,\ldots,\dfrac{p-1}{2}$的排列.

$ \Rightarrow (\dfrac{p-1}{2})!d^{\frac{p-1}{2}}\equiv \prod_{i=1}^{\frac{p-1}{2}}{t_i}\equiv$ 
$\prod_{i=1}^{k}{s_i}\prod_{i=1}^{n}{r_i}\equiv (-1)^n\prod_{i=1}^{k}{s_i}\prod_{i=1}^{n}{(p-r_i)}\equiv (-1)^n(\dfrac{p-1}{2})!$

$ \Rightarrow (\dfrac{d}{p})=d^{\frac{p-1}{2}}\equiv(-1)^n$

特别地,$ d=2$时,对$ 1\le j < \dfrac{p}{4}, 1\le t_j = 2j < \dfrac{p}{2}$;对$ \dfrac{p}{4} < j < \dfrac{p}{2}, \dfrac{p}{2}< t_j = 2j <p, $ 

$ \therefore n = \dfrac{p-1}{2}-[ \dfrac{p}{4}] \Rightarrow ( \dfrac{2}{p}) = (-1)^{\frac{p^2-1}{8}}$

T4: $ x$取遍缩系,则$ x^2$取遍缩系中的一半值.
{\bf 证}: 由T1即得.
\\

$ 4k+1$型素数有无穷多: 假设有穷,考虑$ 4(p_1 p_2 \cdots p_k)^2 + 1$,若为素数则矛盾,若为合数则必有$ 4k+1$型因子.
\\

$ x^4+1$的因子必位$ 8k+1$型: 显然为$ 4k+1$型,又$ 1 = (\dfrac{(x^2+1)^2}{p}) =
(\dfrac{(x^2+1)^2 - (x^4+1)}{p}) = (\dfrac{2x^2}{p}) = (\dfrac{2}{p}) = (-1)^{\frac{p^2-1}{8}}$

推论:$ 8k+1$型素数无穷多,否则考虑$ (2p_1\cdots p_k)^4 +1$

\input{sum_of_square.tex}
\printbibliography

\end{document}


% 冯跃峰 初等数论;潘 初等数论; 小丛书; 奥塞经典; 刘培杰 背景研究;
