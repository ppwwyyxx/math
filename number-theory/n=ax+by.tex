% $File: n=ax+by.tex
% $Date: Thu Mar 08 20:51:30 2012 +0800
% Author: WuYuxin 
\section{表n为ax+by}
$(a,b)=1\Rightarrow \exists x,y\in \mathbb{N}^+,ax-by=1$
\\

$ \forall n>ab-a-b$可表为$ ax+by,x,y\in \mathbb{N}$.

{\bf 证 }:设$ n=a(x_0+bt)+b(y_0-at),$可取$ t$使得$ 0 \le y =y_0-at\le a-1$

则$ ax=n-(y_0-at)b>ab-a-b-(a-1)b=-a\Rightarrow x>-1\Rightarrow x\in \mathbb{N}$
\\

$ n=ab-a-b$不可表.反设结论不成立,则
$ ab=(x+1)a+(y+1)b\Rightarrow b\mid x+1\Rightarrow x+1\ge b$矛盾
\\

写$ n$为$ ax+by,0\le x\le b-1$,若$ n=ax+by$中$ y\ge 0$

则 $ n'=(b-1-x)a+(-1-y)b$中仍有$ 0 \le b-1-x \le b-1$,但$ -1-y <0$.

于是$ n$可表$ \Rightarrow ab-a-b-n$不可表.$ \therefore [0,ab-a-b]$
中有$ \dfrac{(a-1)(b-1)}{2}$个不可表.
\\

在矩形$\begin{matrix} 0\le x\le b\\ 0\le y\le a \end{matrix} $中有$ (a+1)(b+1)$个整点.

其中使$ 0\le ax+by<ab$的整点有$ \dfrac{(a+1)(b+1)}{2} -1$个
\\

$ n=ax+by,x,y\in \mathbb{N^+}$有至少两种表法
$ \Leftrightarrow n$可表为$ ab+a+b+ax+by,x,y\in \mathbb{N}$

i) $ ab+a+b+ax+by=a(1+b+x)+b(1+y)=b(1+a+y)+a(1+x)$

ii) $ ax_1+by_1=ax_2+by_2\Rightarrow a\mid y_2-y_1\Rightarrow y_2\ge a+1$

$ \therefore ax_2+by_2=ab+a+b+(y_2-a-1)b+(x_2-1)a$
