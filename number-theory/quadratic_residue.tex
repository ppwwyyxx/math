% $File: quadratic_residue.tex
% $Date: Thu Aug 02 18:26:01 2012 +0800
% Author: WuYuxin <ppwwyyxxc@gmail.com>
\section{Quadratic Residue}
Def: $ \exists x,x^2\equiv d \pmod p,d<p,p$为奇素数.

T1: $ \pmod p$ 的一个缩系中有$	\dfrac{p-1}{2}$个$ \pmod p$的二次剩余与二次非剩余,
且方程$ x^2 \equiv d \pmod p$若有解必有两解.

{\bf 证}:取绝对最小缩系$ S = \{ -\dfrac{p-1}{2},\cdots -1, 1,\cdots \dfrac{p-1}{2}\}.$ 

$ (\dfrac{d}{p})= 1\Leftrightarrow d \equiv 1^2,\cdots,(\dfrac{p-1}{2})^2 \pmod p. $于是有$ \dfrac{p-1}{2}$个二次剩余
\\

T2(Euler):$ ( \dfrac{d}{p} )\equiv d^{\frac{p-1}{2}} \pmod p$.
(由Fermat定理显然$ d^{\frac{p-1}{2}}\equiv \pm 1 \pmod p$)

{\bf 证:}i)若$ (\dfrac{d}{p})=1,$则$ \exists x_0^2\equiv d\Rightarrow d^{\frac{p-1}{2}}\equiv x_0^{p-1}\equiv 1$

ii)对$ p \nmid d$ ,满足$ ax\equiv d\pmod p$的缩系中的$ a,x$一一对应.

假设$ (\dfrac{d}{p})=-1,$则总有$ a\ne x$,则$ d^{\frac{p-1}{2}}\equiv \prod_{i=1}^{\frac{p-1}{2}}{(a_ix_i)}\equiv (p-1)!\equiv -1\pmod p$
\\

T3(Gauss引理):设对$ 1\le j < \dfrac{p}{2},t_j\equiv jd \pmod p$且$ 0<t_j<p$.
设在$ t_1\ldots,t_{\frac{p-1}{2}}$中有$ n$个$ >\dfrac{p}{2}$,则$ (\dfrac{d}{p})=(-1)^n$

{\bf 证}:设$ >\dfrac{p}{2}$的为$ r_1,\ldots,r_n,<\dfrac{p}{2}$的为$ s_1,\ldots,s_k.k+n=\dfrac{p-1}{2}$.

由于$ \forall 1\le j<i<\dfrac{p}{2},t_j\pm t_i\equiv (j\pm i)d \not \equiv 0\Rightarrow t_j\not\equiv \pm t_i\Rightarrow s_j\not\equiv -r_i\pmod p$

又$ 1\le p-r_i<\dfrac{p}{2},$于是$ s_1,\ldots,s_k,p-r_1,\ldots,p-r_n$为$ 1,2,\ldots,\dfrac{p-1}{2}$的排列.

$ \Rightarrow (\dfrac{p-1}{2})!d^{\frac{p-1}{2}}\equiv \prod_{i=1}^{\frac{p-1}{2}}{t_i}\equiv$ 
$\prod_{i=1}^{k}{s_i}\prod_{i=1}^{n}{r_i}\equiv (-1)^n\prod_{i=1}^{k}{s_i}\prod_{i=1}^{n}{(p-r_i)}\equiv (-1)^n(\dfrac{p-1}{2})!$

$ \Rightarrow (\dfrac{d}{p})=d^{\frac{p-1}{2}}\equiv(-1)^n$

特别地,$ d=2$时,对$ 1\le j < \dfrac{p}{4}, 1\le t_j = 2j < \dfrac{p}{2}$;对$ \dfrac{p}{4} < j < \dfrac{p}{2}, \dfrac{p}{2}< t_j = 2j <p, $ 

$ \therefore n = \dfrac{p-1}{2}-[ \dfrac{p}{4}] \Rightarrow ( \dfrac{2}{p}) = (-1)^{\frac{p^2-1}{8}}$

T4: $ x$取遍缩系,则$ x^2$取遍缩系中的一半值.
{\bf 证}: 由T1即得.
\\

$ 4k+1$型素数有无穷多: 假设有穷,考虑$ 4(p_1 p_2 \cdots p_k)^2 + 1$,若为素数则矛盾,若为合数则必有$ 4k+1$型因子.
\\

$ x^4+1$的因子必位$ 8k+1$型: 显然为$ 4k+1$型,又$ 1 = (\dfrac{(x^2+1)^2}{p}) =
(\dfrac{(x^2+1)^2 - (x^4+1)}{p}) = (\dfrac{2x^2}{p}) = (\dfrac{2}{p}) = (-1)^{\frac{p^2-1}{8}}$

推论:$ 8k+1$型素数无穷多,否则考虑$ (2p_1\cdots p_k)^4 +1$
