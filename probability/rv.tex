% File: rv.tex
% Date: Sat Feb 23 11:35:14 2013 +0800
% Author: Yuxin Wu <ppwwyyxxc@gmail.com>

\section{Random Variables}
\subsection{Definition}
样本空间$ \Omega \rightarrow \mathbb{R}$的函数$ X = X(\omega)$称为\textbf{随机变量}.值域有限或可列称为离散随机变量,值域充满数
轴上的某个区间,称为连续随机变量.记$ F(x) = P(X \le x)$为$ X$的\textbf{分布函数}.

显然,$ F$是$ (-\infty, \infty)$的单调不减函数,有界,于是各点有左右极限,且无穷处有极限.
\begin{equation*}
	\begin{split}
		F(b) - F(c + 0) & = F(b) - \lim \limits_{a \to c^{+}} F(a) =\lim \limits_{a \to c^{+}}P(a < X \le b) \\
		                & = P(\bigcup_{a \to c^{+}}\{a < X \le b\}) = P(c < X \le b)                         \\
		F(d - 0) - F(a) & = P(a < X < d)
	\end{split}
\end{equation*}

即$ F(c + 0) = F(c)(右连续), F(d - 0) = P(X<d), 且可得F(-\infty) = 0, F(\infty) = 1$.

满足以上性质的函数$ F$必定为某随机变量的分布函数.

任意Borel集$ B\subset \mathbb{R}, P(X \in B) 可由F$ 计算得到.

特别的,$ P(X=x) = F(x) - F(x - 0),若F在x连续,则P(X=x) = 0$
\\

\textbf{$p$-分位数:}满足$ P(X \le x) \ge p, P(X <x) \le p 的x. $

对连续随机变量,等价定义$F(x) = p$的点为$p$-(下侧)分位数.

$ p =\dfrac{1}{2}$时称为\textbf{中位数}.$\Leftrightarrow \dfrac{1}{2} \le F(x) \le \dfrac{1}{2} + P(X=x) $

中位数的统计意义:使得$\Ex{|X-a|}$最小的$a$
\\

存在非负可积函数,使得$ F(x) = \int_{-\infty}^{x}{p(t)dt}$,则称$ p(x)$为$ X$的\textbf{概率密度函数}.

在$ F(x)$导数存在的点有$ p(x) = F'(x)$,其余点处$ p(x)$可任意取值.
\\

\textbf{密度公式:}
函数$g(x)单调连续,h(x) = g^{-1}(x)连续可微,则Y = g(X)$的密度函数为$ p_Y(y) = p_X(h(y))|h'(y)|$

\proof:$F_Y(y) = P(g(X)\le y)=\int_{h(-\infty,y]}{p_X(x)\mathrm{d}x}
\xlongequal{x=h(z)}\int_{-\infty}^{y}RHS$

应用: $X\sim \Gamma(\alpha, \lambda) \Rightarrow  kX \sim \Gamma(\alpha, \dfrac{\lambda}{k}), 2\lambda X \sim \chi^2(2\alpha)$
\\

\textbf{Diagonal:}$ F_X(x)$若为严格增的连续函数,$F_X(X) \sim U(0,1)$

\proof:$ F_{F_X(X)}(x) = P(F_X(X)\le x) = P(X \le F_X^{-1}(x)) = F_X(F_X^{-1}(x)) =x$

\subsection{Expectation \& Variance}
\begin{align*}
    \E{X} & = \int_{\mathbb{R}}{x\mathrm{d}F(x)},其中积分为Lebsegue积分. \\
			& = \left\{  \begin{matrix}
	\sum{x_ip(x_i)} = \sum_{n=0}^{\infty}[{P(X>n) - P(X<-n)}], 离散,\sum{|x_i|}P(X = x_i)< \infty \\
	\int_{\mathbb{R}}{xp(x)\mathrm{d}x} = \int_{\mathbb{R^+}}{P(X>x)\mathrm{d}x} - \int_{\mathbb{R^-}}{P(X<x)\mathrm{d}x},连续,\int_{\mathbb{R}}{|x|p(x)\mathrm{d}x} < \infty
\end{matrix}\right.
 \end{align*}

$\bullet$ 绝对收敛保证了和的存在且与顺序无关.

特别地,当$ X>0$时,有期望计算公式$ \E{X} = \sum_{n=0}^\infty{P(X>n)}$. 本质是换一种切分方式求面积.
\\

当期望存在时: $nP(X > n) = n\int_n^{\infty}\mathrm{d}F(x)\le \int_n^{\infty}x\mathrm{d}F(x)$

上式取极限$n\to \infty, 得 nP(X>n) \to 0 ,即 \lim \limits_{x\to \infty}x(1-F(x)) = 0$

同理有$\lim \limits_{x \to -\infty}xF(x) = \lim \limits_{x\to \infty}x(1-F(x)) = 0$

由此极限可推出\textbf{期望的几何意义:} $\int_{-\infty}^{\Ex{X}}{F(x)\mathrm{d}x}
= \int_{\Ex{X}}^{\infty}{(1-F(x))\mathrm{d}x}$

即:$y=F(x), x=\Ex{X}, 将0\le y \le 1$分成面积相等的两部分.

\proof:对两边进行分部积分即可.
\\

\textbf{Cauchy-Schwartz:}

$\E{X^2},\E{Y^2} < \infty, 则(\E{XY})^2 \le \E{X^2}\E{Y^2}$

\proof:考虑$ f(u) = \E{Xu+Y}^2 = (\E{X^2})u^2 + 2\E{XY}u+\E{Y^2}$的判别式即可.
\\

\textbf{期望的统计意义:}

$\E{(X-a)^2} =\E{(X-\Ex{X})^2} + 2 \E{(X-\Ex{X})(\Ex{X}-a)} + (\Ex{X}-a)^2=
\E{(X-\Ex{X})^2} +
(\Ex{X}-a)^2 \ge \E{(X-\Ex{X})^2}. $
\\

若$ \E{X^2}$存在,则定义$ \Var{X} = \E{X - \Ex{X}}^2 = \left \{
  \begin{matrix}\sum(x_i - \Ex{X})^2p(x_i) \\ \int_{\mathbb{R}}{(x-\Ex{X})^2p(x)\mathrm{d}x} \end{matrix}\right.$

$ \Var{X} = \E{X^2 - 2X\Ex{X} + (\Ex{X})^2} = \Ex{X^2} - (\Ex{X})^2$
\\

\textbf{其他统计量:}

变异系数$ C_v(X) = \dfrac{\sigma(X)}{\Ex{X}} = \dfrac{\sqrt{\Var{X}}}{\Ex{X}}$.消去了量纲的影响.

偏度系数$ \beta_{s} = \dfrac{\E{X-\Ex{X}}^3}{(\Var{X})^{\frac{3}{2}}}$.描述偏离对称性的程度.

峰度系数$ \beta_{k} = \dfrac{\E{X-\Ex{X}}^4}{(\Var{X})^2} - 3$.描述相比于正态分布的尖峭程度(尾部粗细).正态分布峰度为0.

\subsection{Probability Generating Function}
对一取非负整数的随机变量$ X$,设$ p_k = P(X=k)$,定义$ G_X(t) = \E{t^X} = \sum_{k=0}^\infty{p_kt^k}$

注意到$  p_k = \dfrac{G^{(k)}(0)}{k!}, G(1) = 1, G(t)对|t|\le 1$绝对收敛.

$G(t) $在$ [0,1]$上连续,此时由$ G'(t) = \sum{kp_kx^{k-1}}, G''(t) = \sum{k(k-1)p_kx^{k-2}}$知,$ G$是递增的下凸函数.

$ G^{(r)}(1) = \E{X^{\underline{r}}}, $其中$ x^{\underline{r}}$为$ x$的$ r$次降阶乘.

$\Rightarrow \Var{X} = G''(1) - G'(1)^2 + G'(1) $
\\

$ X,Y$独立$ \Rightarrow G_{X+Y}(t) = G_X(t)G_Y(t), G_{X-Y}(t) = G_X(t)G_Y(t^{-1})$

\textbf{Compound Distribution: } $ X_i $ i.i.d,$ N$是随机变量,$ S_N = \sum_{k=1}^N{X_i} $

$G_{S_N}(t) = \sum_{n=0}^\infty{\E{t^{S_N} | N=n}P(N=n)} =\sum_{n=0}^\infty{\E{t^{S_n}}P(N=n)} = \sum_{n=0}^\infty{\E{t^X}^n P(N=n)} = G_N(G_X(t)) $

$ \Rightarrow \E{S_N} = \E{N}\E{X}, \Var{S_N} = \E{N}\Var{X} + \Var{N}\E{X}^2$

\subsection{Characteristic Function}
$ \varphi_X(t) = \E{e^{itX}}$

$ \varphi_{\vec{X}}(\vec{t}) = \E{e^{i\vec{t}^T\vec{X}}}$

\begin{description}

  \item[界:]$ |\varphi_X(t)| = |\E{e^{itX}}| \le \E{|e^{itX}|} = 1$

  \item[对称性:] $ \varphi_X(t)是实偶函数 \Leftrightarrow \varphi_X(t) =
      \overline{\varphi_X(t)} = \E{\overline{e^{itX}}}=\E{e^{-itX}} = \varphi_X(-t) = \varphi_{-X}(t)
      \Leftrightarrow F_X(x) = F_{-X}(x)为对称分布\Leftrightarrow F_X(x)关于(0,\dfrac{1}{2})对称$

        \item[线性变换:] $ \varphi_{a+bX}(t) = e^{iat}\varphi_X(bt)$

        \item[卷积变乘积:] $ X,Y$独立,$ \varphi_{X+Y}(t) = \E{e^{it(X+Y)}} = \varphi_X(t)\varphi_Y(t)$

          \item[矩的计算:] $ \varphi^{(k)}(t) =
            \int_{\mathbb{R}}i^kx^ke^{itx}p(x)\mathrm{d}x =
            i^k\E{X^ke^{itX}}\Rightarrow \varphi^{(k)}(0) = i^k\Ex{X^k}$

          \item[独立性判定:] $ X_1,\cdots ,X_n独立\Leftrightarrow \varphi_{\vec{X}}(\vec{t})=\prod_{k=1}^n{\varphi_{X_k}(t_k)}$

          \item[分析性质:]\hfill\\
   $ \varphi(x)在\mathbb{R}上一致连续$

对连续型随机变量$ X, \varphi_X(t) = \int_{\mathbb{R}}e^{itx}p(x)\mathrm{d}x$为$ p(x)$的Fourier变换.

于是有逆变换: $F(x) = \lim \limits_{y \to - \infty} \lim \limits_{T\to \infty}\dfrac{1}{2\pi}\int_{-T}^T{\dfrac{e^{-ity}-e^{-itx}}{it}\varphi(t)\mathrm{d}t} $

$\hspace{2.3cm} p(x) = \dfrac{1}{2\pi}\int_{\mathbb{R}}e^{-itx}\varphi_X(t)\mathrm{d}t$

  $即 F(x)与\varphi(t)有一一对应的关系. 且连续性定理:$
  \[ \{ F_n(x)\}弱收敛到F(x)\Leftrightarrow \{ \varphi_n(x)\}收敛到\varphi(x)\]
表明这种对应关系也是连续的
      \end{description}

