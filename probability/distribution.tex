% File: distribution.tex
% Date: Fri Feb 22 00:13:29 2013 +0800
% Author: Yuxin Wu <ppwwyyxxc@gmail.com>
\section{Distributions}

\subsection{Common Discrete Distributions}
\begin{enumerate}
  \item \textbf{Bernoulli (Binomial):}$ b(n,p)$

    $ P(X = k) = C_n^kp^k(1-p)^{n-k}, k = 0,\cdots,n$

    $ \Ex{X} = np, \Var{X} = \E{X(X-1)}+\Ex{X}-(\Ex{X})^2 = np(1-p)$

    $	b(n,p)  = n * b(1,p)\Rightarrow \varphi(t) = (1-p+pe^{it})^n$

  概率最大值发生在$ k = \left \{ \begin{matrix}
      (n+1)p , (n+1)p - 1 &, (n+1)p \in \mathbb{N} \\
      \lfloor(n+1)p\rfloor&,  (n+1)p \not \in \mathbb{N}
    \end{matrix}\right.$

\item \textbf{Poisson:} $ P(\lambda),(\lambda > 0)$

  $ P(X = k) = \dfrac{\lambda^k}{k!}e^{-\lambda}$

  概率最大值发生在$ k = \lfloor \lambda \rfloor$

  $ \Ex{X} = \Var{X} = \lambda, \varphi(t) = \sum_{k=0}^{\infty}e^{itk}\dfrac{\lambda^k}{k!}e^{-\lambda} = e^{\lambda(e^{it} - 1)}$

  \textbf{Poisson 定理},对二项分布$ b(n,p_n),\lim \limits_{n\to \infty} np_n = \lambda > 0$,
  \begin{flalign*}
    P(X = k) & = C_n^kp_n^k(1-p_n)^{n-k}                                                                                \\
    & =\dfrac{n(n-1)\cdots(n-k+1)}{k!n^k}(np_n)^k(1-\dfrac{np_n}{n})^{n-k}                                     \\
    & =(1-\dfrac{1}{n})\cdots(1-\dfrac{k-1}{n})\dfrac{[\lambda + o(1)]^k}{k!}[1-\dfrac{\lambda+o(1)}{n}]^{n-k} \\
    & \rightarrow \dfrac{\lambda^k}{k!}e^{-\lambda}, n\to \infty
\end{flalign*}

    \item \textbf{Geometry:}$ G(p)$

      $ P(X=k) = p(1-p)^{k-1}, k \in \mathbb{N^+}$

      $ \Ex{X} = \sum_{n=1}^{\infty}\sum_{k=1}^{n}{p(1-p)^{n-1}}=
      \sum_{k=1}^{\infty}\sum_{n=k}^{\infty}{p(1-p)^{n-1}}= \dfrac{1}{p}$

      类似方法使用两次求出$ \E{X(X-1)}\Rightarrow \Var{X} = \dfrac{1-p}{p^2}$

      $ \varphi(t) = \dfrac{pe^{it}}{1-(1-p)e^{it}}$

      尾概率$ P(X>m) = (1-p)^m$

      无记忆性 $\Leftrightarrow P(X>m+n) = P(X>m)P(X>n)\Leftrightarrow X\sim G(P(X\le 1)) $

          (即解Cauchy方程)

        \item \textbf{HyperGeometry:} $ h(n,N,M),(n, M\le N)$

          意义:$ N$件物品含有$ M$件次品,不放回抽取$ n$次得到的次品数.

          $ P(X=k) = \dfrac{C_M^kC_{N-M}^{n-k}}{C_N^n}, k \in [\max\{0, n-N + M\}, \min\{M,n\}]$

          $ \Ex{X} = \sum_{k}{k\dfrac{C_M^kC_{N-M}^{n-k}}{C_N^n}}  = \dfrac{Mn}{N}\sum_{k}{\dfrac{C_{M-1}^{k-1}C_{N-M}^{n-k}}{C_{N-1}^{n-1}}} = \dfrac{Mn}{N}$

          (注意到每次的期望都是$ \dfrac{M}{N}$即可)

          类似地使用Vandermonde Convolution,有$ 	\E{X(X-1)} = \dfrac{M(M-1)n(n-1)}{N(N-1)}$

          $\Rightarrow \Var{x}  = \dfrac{nM(N-M)(N-n)}{N^2(N-1)} $

        二项逼近:当$N\to\infty, \frac{M}{N}\to p $时,$ h(n,N,M)\to b(N,p)$

      \item \textbf{Pascal (Negative Binomial):} $ Nb(r,p)$

        意义:事件发生第$ r$次时的实验次数.$ Nb(r, p) = r * G(p) $

        $ P(X = k) = C_{k-1}^{r-1}p^r(1-p)^{k-r}, k = r,r+1,\cdots$

        $ \Ex{X} = \dfrac{r}{p}, \Var{x} = \dfrac{r(1-p)}{p^2}$

        $ \varphi(t) = (\dfrac{pe^{it}}{1-(1-p)e^{it}})^r$

        \textbf{Banach's matchbox} problem:

        左右各$ n$根火柴随机用,发现一边用完时另一边还还有$ k$根的概率$ p_{n,k}$:

        事件(的一半)相当于:成功$ n+1$次时恰已有$ n-k$次失败,设$ Y\sim Nb(n+1, \dfrac{1}{2})$

        立即有:$ p_{n,k} = 2P(Y = 2n-k+1) = C_{2n-k}^{n}2^{-2n+k}$

      \item \textbf{Fixed Points}

        See my paper at \url{http://learn.tsinghua.edu.cn:8080/2011011271/fixed_points.pdf}

    \end{enumerate}
    \subsection{Common Continuous Distributions}
    \begin{enumerate}
      \item \textbf{Uniformed:} $U(a,b)$

        $p(x) = \dfrac{1}{b-a}I_{[a,b]}(x) = \left \{ \begin{matrix}\dfrac{1}{b-a}, & a\le x \le b \\0, &  \text{else}\end{matrix}\right.$
        其中$ I$为区间示性函数.

        $\Ex{X} = \dfrac{a+b}{2}, \Var{X} = \dfrac{(b-a)^2}{12}$

        $ U(-1,1)$的特征函数$\varphi_{X'}(t)=\int_{0}^1{\cos(tx)\mathrm{d}x} = \dfrac{\sin t}{t}\Rightarrow \varphi(t) =
      e^{\frac{(a+b)it}{2}}\dfrac{\sin(\frac{(b-a)t}{2})}{\frac{(b-a)t}{2}}$

      或:$ \varphi(t) = \int_{a}^b{\dfrac{e^{itx}}{b-a}\mathrm{d}x} = \dfrac{e^{ibt}-e^{iat}}{it(b-a)}$,与上式其实是相等的.

    \item \textbf{Exponential:}$Exp(\lambda),(\lambda > 0)$

      $p(x) = \lambda e^{-\lambda x}I_{\mathbb{R^+}}(x)$

      $F(x) = 1-e^{-\lambda x}$

      设备在时刻$t$的失效率$\lambda(t) = \lambda$为常数,则寿命$X\sim Exp(\lambda)$

      $\Ex{X} = \int_{\mathbb{R^+}}{P(X>x)\mathrm{d}x} = \int_{\mathbb{R^+}}{e^{-\lambda x}\mathrm{d}x} = \dfrac{1}{\lambda}$

      $ \Ex{X^2} = \int_{\mathbb{R^+}}{P(X^2 > x)\mathrm{d}x} \xlongequal{x = u^2} \dfrac{2}{\lambda}\int_{\mathbb{R^+}}{u\lambda e^{-\lambda u}\mathrm{d}u}
      = \dfrac{2\Ex{X}}{\lambda} = \dfrac{2}{\lambda^2}$

      $\Var{X} = \Ex{X^2} - (\Ex{X})^2 = \dfrac{1}{\lambda^2}$

      $ \varphi(t) = \lambda\int_{\mathbb{R}^+}e^{itx-\lambda x}\mathrm{d}x=\dfrac{\lambda}{\lambda-it}$

      无记忆性 $\Leftrightarrow P(X>m+n) = P(X>m)P(X>n)\Leftrightarrow X\sim Exp(\lambda)$

          $(0,t]$发生的次数$\sim P(\lambda t),$则第一次发生的时间$\sim Exp(\lambda)$

          $ X\sim U(0,1)\Rightarrow  \dfrac{-\ln X}{\lambda}\sim Exp(\lambda)$

        $ X,Y\sim Exp(\frac{1}{\lambda})\Rightarrow X-Y\sim L(0,\lambda), |X-Y|\sim Exp(\dfrac{1}{\lambda})$


    \item \textbf{Gauss (Normal):}$N(\mu, \sigma^2)$

      $p(x) = \dfrac{1}{\sqrt{2\pi \sigma^2}}e^{-\frac{(x-\mu)^2}{2\sigma^2}}$

      对$Y\sim N(\mu, \sigma^2), $标准化:$X = \dfrac{Y - \mu}{\sigma},则 X\sim N(0,1)$

      $ P(X \le x) = \int_{-\infty}^{x}{\frac{1}{\sqrt{2\pi}}e^{-\frac{u^2}{2}}\mathrm{d}u} \xlongequal{def}\Phi(x)$

      利用Poisson积分,可验证$\Phi(\infty) = 1$

      $ P(|Y-\mu| < k\sigma) = \Phi(k) - \Phi(-k) = 2\Phi(k)-1$
      \\

      $\Ex{X} = \int_{\mathbb{R}}{x\dfrac{1}{\sqrt{2\pi}}e^{-\frac{x^2}{2}}\mathrm{d}x} = 0$

      $ \Var{X} = \Ex{X^2} = 1. 分部积分$

      $ \Ex{X^{2k+1}} =0, \Ex{X^{2k}} = (2k-1)!!$

      $\varphi_X(t) = 2\int_{\mathbb{R}^+}\cos(tx)\dfrac{1}{\sqrt{2\pi}}e^{-\frac{x^2}{2}}\mathrm{d}x$,

      求导有$ \dfrac{\mathrm{d}\varphi_X(t)}{\mathrm{d}t} = 2\int_{\mathbb{R}^+}\sin(tx)\mathrm{d}(\dfrac{1}{\sqrt{2\pi}}e^{-\frac{x^2}{2}})
      = -t\varphi_X(t)\Rightarrow \varphi_X(t) = e^{-\frac{t^2}{2}}$.

    于是,$ \varphi_Y(t) = e^{i\mu t-\frac{\sigma^2t^2}{2}}$
    \\

    误差函数:$ Erf(x) = 2\Phi(x\sqrt{2})-1$

  \item \textbf{Gamma:} $\Gamma(\alpha, \lambda),(\alpha > 0为形状参数,\lambda>0为尺度参数)$

    $ \Gamma 函数: \Gamma(x) = \int_{0}^{\infty}t^{x-1}e^{-t}\mathrm{d}t$

    $ p(x) = \dfrac{\lambda^{\alpha}}{\Gamma(\alpha)}x^{\alpha - 1}e^{-\lambda x}I_{[0,\infty)}(x)$

    $ \Ex{X} = \dfrac{\lambda^{\alpha}}{\Gamma(\alpha)}\Gamma(\alpha+1) = \dfrac{\alpha}{\lambda}$

    $ \Ex{X^2} = \dfrac{\alpha(\alpha+1)}{\lambda^2},\Var{X} = \dfrac{\alpha}{\lambda^2}$

    $ \Gamma(1,\lambda )= Exp(\lambda), \Gamma(\dfrac{n}{2},\dfrac{1}{2}) = \chi ^2(n)$

    $ \varphi(t) = (\dfrac{\lambda-it}{\lambda})^{\alpha}$
  \item \textbf{Beta:} $ Be(a,b),(a,b>0)$

    $ Beta函数:B(a,b) = \int_{0}^{1}{x^{a-1}(1-x)^{b-1}\mathrm{d}x} = \dfrac{\Gamma(a)\Gamma(b)}{\Gamma(a+b)}$

    $ p(x) = \dfrac{1}{B(a,b)}x^{a-1}(1-x)^{b-1}I_{(0,1)}(x)$

    $ \Ex{X} = \dfrac{\Gamma(a+b)}{\Gamma(a)\Gamma(b)}\dfrac{\Gamma(a+1)\Gamma(b)}{\Gamma(a+b+1)} = \dfrac{a}{a+b}$

    $ \Ex{X^2} = \dfrac{\Gamma(a+b)}{\Gamma(a)\Gamma(b)}\dfrac{\Gamma(a+2)\Gamma(b)}{\Gamma(a+b+2)} = \dfrac{a(a+1)}{(a+b)(a+b+1)}$

    $ \Var{X} = \dfrac{ab}{(a+b)^2(a+b+1)}$

    特例: $p(x)=nx^{n-1}\Leftrightarrow X\sim Be(n,1);\; p(x) = \dfrac{x^{n-1}(1-x)}{n+n^2}\Leftrightarrow X\sim Be(n,2)$


    \item \textbf{Chi Square:} $ \chi^2(k).(自由度k>0)$

      $ Y_1 \xlongequal{i.i.d} \cdots \xlongequal{i.i.d} Y_k \sim N(0,1)
    \Rightarrow X = \sum{Y^2}\sim \chi^2(k)$

    $ p(x) = \dfrac{x^{\frac{k}{2}-1}e^{-\frac{x}{2}}}{2^{k/2}\Gamma(\frac{k}{2})} I_{[0,+\infty)(x)}$

    $ \Ex{X} = k. \Var{X} = 2k$

    $ \varphi(t) = (1-2it)^{-\frac{n}{2}}$

  \item \textbf{Logarithmic Normal:}$ LN(\mu, \sigma^2)$

    $ X\sim N(\mu,\sigma^2)\Rightarrow Y = e^{X}\sim LN(\mu,\sigma^2)$

  $ p(x) = \dfrac{1}{\sqrt{2\pi}x\sigma}e^{-\frac{(\ln x- \mu)^2}{2\sigma^2}}I_{\mathbb{R}^{+}}(x)$

  $ \Ex{X} = e^{\mu + \frac{\sigma^2}{2}}, \Var{X} = e^{2\mu+\sigma^2}(e^{\sigma^2}-1). 中位数e^\mu$

\item \textbf{Cauchy:}$ Cauchy(\mu, \lambda).$

  $ Y \xlongequal{i.i.d} Z \sim N(0,1) \Rightarrow \dfrac{Y}{Z}\sim Cauchy(0,1)$

$ p(x) = \dfrac{\lambda}{\pi(\lambda^2 +(x-\mu)^2)}$

期望与方差不存在.

$ \varphi(t) = e^{i\mu t - \lambda |t|}$

 \item \textbf{Fisher-Snedecor:} $ F(m, n)$

   独立的$X\sim \chi^2(m),Y\sim\chi^2(n).F=\dfrac{X/m}{Y/n}\sim F(m,n).$

   $ p_{F}(x) =
   \dfrac{1}{B(\frac{m}{2},\frac{n}{2})}(\dfrac{m}{n})^{\frac{m}{2}}x^{\frac{m}{2}-1}(1+\dfrac{mx}{n})^{-\frac{m+n}{2}} I_{[0,+\infty)}$

   $ \Ex{X} = \dfrac{n}{n-2}(n>2), \Var{X} = \dfrac{2n^2(m+n-2)}{m(n-2)^2(n-4)}(n>4)$

   $ \dfrac{mF}{n+mF} \sim B(\dfrac{m}{2},\dfrac{n}{2})$

   $ X\sim F(k_1,k_2) \Leftrightarrow \dfrac{1}{X}\sim F(k_2,k_1) , F_{1-a}(k_1,k_2) = \dfrac{1}{F_a(k_2,k_1)}$

     注意到$ \dfrac{1}{m}\chi^2(m)$是$ m+1$个样本的样本方差.

   \item \textbf{Student t:} $ t(n)$. Student是其论文笔名(1908)

     独立的$X\sim N(0,1),Y\sim\chi^2(n).t=\dfrac{X}{\sqrt{\frac{Y}{n}}}\sim t(n) $

     由对称性,$ F_t(x)=F_t(0)+\dfrac{1}{2}P(t^2\le x^2)=\dfrac{1}{2}+\dfrac{1}{2}F_{t^2}(x^2)$

     而$ t^2\sim F(1,n).$于是$ p_t(x)=\dfrac{\Gamma(\frac{n+1}{2})}{\sqrt{n\pi}\Gamma(\frac{n}{2})}(1+\dfrac{x^2}{n})^{-\frac{n+1}{2}}$

     $t(1) = Cauchy(0,1) ,t(\infty)=N(0,1)$

     $ \Ex{X}=0(n>1), \Var{X}=\dfrac{n}{n-2}(n>2)$

     注意到对于正态样本,有$ \dfrac{\sqrt{n}(\bar{x}-\mu)}{\sigma}\sim N(0,1)$

     $ \Rightarrow \dfrac{\sqrt{n}(\bar{x}-\mu)}{s} =
   \dfrac{N(0,1)}{\sqrt{\dfrac{(n-1)s^2}{\sigma^2(n-1)}}} =
   \dfrac{N(0,1)}{\sqrt{\dfrac{\chi^2(n-1)}{n-1}}} \sim t(n-1)$
 \item \textbf{Weibull:} $ W(\lambda,k)$

   设备失效率=$ \lambda t^{k-1}\Rightarrow  寿命服从W(\lambda,k). $

 $ W(\lambda,1) = Exp(\lambda)$

 $ p(x) = \dfrac{k}{\lambda}(\dfrac{x}{\lambda})^{k-1}e^{-(\frac{x}{\lambda})^k}I_{[0,+\infty)}(x)$

 $ F(x) = 1-e^{-(\frac{x}{\lambda})^k}$

 $ \Ex{X} = \lambda \Gamma(1+\dfrac{1}{k}), \Var{X} =
 \lambda^2\Gamma(1+\dfrac{2}{k})-(\Ex{X})^2$

\item \textbf{Rayleigh:} $ R(\sigma)$

  $ X\xlongequal{\text{i.i.d}}Y\sim N(0, \sigma^2)\Rightarrow \sqrt{X^2+Y^2} \sim R(\sigma)$

$ p(x) = \dfrac{x}{\sigma^2}e^{-\frac{x^2}{2\sigma^2}}I_{[0,+\infty)}$

$ F(x) = 1-e^{-\frac{x^2}{\sigma^2}}$

$ \Ex{X} = \sigma\sqrt{\dfrac{\pi}{2}}, \Var{X} = \dfrac{4-\pi}{2}\sigma^2$

\item \textbf{Laplace:} $ L(\mu,b)$

  $ p(x)=\dfrac{1}{2b}e^{-\frac{|x-\mu|}{b}}, F(x)=\dfrac{1+\sgn(x-\mu)}{2}(1-e^{-\frac{|x-\mu|}{b}})$

  $ \Ex{X} = \mu, \Var{X} = 2b^2$

  $ \varphi(x) = \dfrac{e^{\mu it}}{1+b^2t^2}$


\item \textbf{Pareto}
  $ p(x) = (\alpha-1)x_0^{\alpha-1}x^{-\alpha}I_{[x_0,+\infty]}, \alpha>2, x_0>0$

  $ \Ex{X} = \dfrac{\alpha-1}{\alpha-2}x_0, \Var{X} = \dfrac{(\alpha-1)x_0^2}{(\alpha-3)(\alpha-2)^2}(\alpha>3)$

  \end{enumerate}

