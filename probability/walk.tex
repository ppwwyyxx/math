% File: walk.tex
% Date: Sun Feb 24 14:12:08 2013 +0800
% Author: Yuxin Wu <ppwwyyxxc@gmail.com>

\section{Random Walk}
\subsection{Combinatorial Perspectives}
\label{sec:walk}
\textbf{Counting Paths}:从$ (0,0)走到 (n,x),(|x|\le n, n + x\equiv 0 \pmod 2)$的折线条数:

  设有$ p$个$ +1,q个-1,\begin{cases}n=p+q\\x=p-q \end{cases}$,
  条数为$ N_{n,x} = C_{p+q}^p = C_{n}^{\frac{n+x}{2}}$
  \\

  \textbf{Reflection Principle}:

  设$ b_0,b_n>0,从(0,b_0)到(n,b_n)$的经过$ x$轴的折线与$(0,-b_0)$到$(n,b_n)$的折线一一对应.
  \\

  \textbf{Ballot Theorem}:甲$ p>$乙$ q$票,甲始终领先:

  等价于$ (1,1)$到$ (p+q,p-q)$的不过$ x$轴的折线,有$ N_{n-1,x-1} -
  N_{n-1,x+1}=\dfrac{x}{n}N_{n,x}$
  \\

  下面只考虑从原点出发的折线,采用如下记号:
  $\; X_k = \pm 1, S_k = \sum_{i=1}^k{X_i},S_0=0$

  \textbf{Reach:}$\; p_{n,r} = \Pr{S_n=r} = N_{n,r}2^{-n} = C_{n}^{\frac{n+r}{2}}2^{-n}$

  \textbf{Return}:$\; u_{2n} = p_{2n,0} = \dfrac{(2n)!}{(n!)^2}2^{-2n}\approx \dfrac{1}{\sqrt{\pi n}}$ (Stirling's Approximation)
  \vspace{0.6cm}

  \textbf{Important Lemma}:

  $ \Pr{S_1,\cdots ,S_{2n}\neq 0} = \Pr{S_{2n}=0}=2\Pr{S_1,\cdots ,S_{2n}>0} =
    \Pr{S_1,\cdots ,S_{2n}\ge 0}$
\begin{align*}
  \proof:
   \Pr{S_1,\cdots ,S_{2n}>0} &= \sum_{r=1}^{n}{\Pr{S_1 ,\cdots S_{2n-1}>0,S_{2n}=2r}}\\
   &=\sum_{r=1}^{n}{\dfrac{N_{2n-1,2r-1}-N_{2n-1,2r+1}}{2^{2n}}} \; \text{(by ballot theorem)}\\
   &=\sum_{r=1}^n{\dfrac{1}{2}(p_{2n-1,2r-1}-p_{2n-1,2r+1})} \\
   &=\dfrac{1}{2}p_{2n-1,1}-0 = \dfrac{1}{2}u_{2n}
\end{align*}
又由$ S$的整数连续性知$ \Pr{S_1,\cdots ,S_{2n}\neq0}=2\Pr{S_1,\cdots ,S_{2n}\gtrless 0}$

考虑每一条恒正路径的概率,第一步走到$ (1,1)$的概率为$ \frac{1}{2},$之后对应于一条长为$ 2n-1$的恒非负路径,即
$ \Pr{S_1,\cdots ,S_{2n}>0}=\dfrac{1}{2}\Pr{S_1,\cdots ,S_{2n-1}\ge 0} = \dfrac{1}{2}\Pr{S_1,\cdots ,S_{2n}\ge 0}$

最后一步是因为$ S_{2n-1}\ge 0$蕴含$ S_{2n}\ge 0. $\qed
\\

\textit{几何证明}:考虑到$ S_{2n}=0$的每一条路径,设其所有最小值点中最左边的一个为$ M(k,m),$将$ (0,0)\sim M$的部分沿$ x=k$翻转后续接到
$ (2n,0)$后,以$ M$为新原点,便得到一条恒非负路径.\qed
\vspace{0.6cm}

  \textbf{First Return}:$ f_{2n} = \Pr{S_1,\cdots ,S_{2n-1}\neq 0, S_{2n} = 0}$

  由引理立刻得,$ f_{2n} = u_{2n-2}-u_{2n} = \dfrac{u_{2n}}{2n-1}$

  有递推关系:$ u_{2n}=\sum_{i=1}^n{f_{2i}u_{2n-2i}}$
  \vspace{0.6cm}

  \textbf{Last Return:}长为$ 2n$的路径,最后一次经过$ x$轴在$ 2k$处的概率为$ a_{2n,2k} = u_{2k}u_{2n-2k}$.

  \proof:由引理,前$ 2k$次结束于$ x$轴的概率为$ u_{2k}$,后$ 2n-2k$次不经过$ x$轴的概率为$ u_{2n-2k}$.

  \textbf{ArcSin Distribution:}由$ u_{2k}$的逼近,有$ a_{2n,2k}\approx \dfrac{f(\frac{k}{n})}{n},f(x)=\dfrac{1}{\pi\sqrt{x(1-x)}}$

  积分:$\sum_{k<xn}a_{2n,2k}\approx \dfrac{2}{\pi}\arcsin{\sqrt{x}}$
\vspace{0.6cm}

\textbf{Sojourn Time}:有$ 2k$时间在一象限,$ 2n-2k$时间在四象限的概率为$ b_{2n,2k} = a_{2n,2k}$

\proof:设First Return发生在$ 2r$处,此前恒正有$ \frac{2^{2r}f_{2r}}{2} \times 2^{2n-2r}b_{2n-2r,2k-2r}$种可能,恒负有
$ \frac{2^{2r}f_{2r}}{2}\times 2^{2n-2r}b_{2n-2r,2k}$种可能.

$ \Rightarrow b_{2n,2k} = \dfrac{f_{2r}}{2}[\sum_{r=1}^k{b_{2n-2r,2k-2r}} + \sum_{r=1}^{n-k}{b_{2n-2r,2k}}],(1\le k \le n-1)$

再由$ b_{2n,0}=b_{2n,2n}=u_{2n}$,归纳即证.
\vspace{0.6cm}

\textbf{Changes of Sign}:显然$ S_{2n+1}\neq 0$,在前$ 2n+1$次中符号变化的次数为$ r$的概率为$ \xi_{2n+1,r} $

对$ r>1,$只考虑$ X_1=1$的情形,将路径在变号点分两段计数归纳,可证$ \xi_{2n+1,r}=2p_{2n+1,2r+1}.$

领先不易改变:$ \xi_{n,0}>\xi_{n,1}>\cdots $
\vspace{0.6cm}

\textbf{Maximum}:由反射原理,$ (0,0)\sim (n,k)$的路径,最大值为$ r$的概率为$ p_{n,2r-k}-p_{n,2r+2-k}$

上式对$ k$求和,得长为$ n$的路径,最大值为$ r$的概率为$ p_{n,r}或p_{n,r+1}$(只有一者有意义).

取$ r=0$即是引理的证明.
\vspace{0.6cm}

\textbf{First Passage}:$ n$时刻初过$ r$的概率为$\varphi_{n,r}$

$(0,0)\sim (n-1,r-1) $的最大值为$ r-1$的路径概率为$ p_{n-1,r-1}-p_{n-1,r+1}$

$ \Rightarrow \varphi_{n,r} = \dfrac{p_{n-1,r-1}-p_{n-1,r+1}}{2} = \dfrac{rp_{n,r}}{n}$
\vspace{0.6cm}

\textbf{$ \mathbf{r}$th Return}:第$ r$次返回发生在$ n$时刻的概率为$ \varphi_{n-r,r}$

\proof:设第$ r$次返回发生在时刻$ n$的长为$ n$的路径有$ A$条,则其中全在$ x$轴下方的路径有$ \dfrac{A}{2^r}$条.
对每一条这种路径,将它所有从$ x$轴出发的那$ r$步删去,对应到一条$ n-r$时刻初过$ r$的路径.
\vspace{0.6cm}

\textbf{Sojourn Time with Return:}设$ S_{2n}=0$,前$ 2n$时间中,$ 2k$时间在上方的概率与$ k$无关.

\proof:设$ 2r$时刻First Return,讨论$ 1\sim 2r$恒正或恒负,和式变换消去$ k$.概率为$ \dfrac{u_{2n}}{n+1}=2f_{2n+2}$
\vspace{0.6cm}

\textbf{Duality}:设$ X_i^\star = X_{n-i},S_{i}^\star = S_n-S_{n-i}$,可得到一系列对偶命题.起点-终点,初-末对应.
\vspace{0.6cm}

\textbf{Asymptotic}:

访问次数$ V_n,$部分和$ S_n$的阶为$ \sqrt{n}.$

$  P(S_n < t\sqrt{n})\to \Phi(t), P(V_n < t\sqrt{n}) \to 2\Phi(2t) -1$
\\

初过$ r$时间与第$ r$次返回时间的阶为$ r^2$.

初过$ r$发生在$ tr^2$之前的概率$ \to 2 - 2 \Phi(\dfrac{1}{\sqrt{t}})$

第$ r$次返回发生在$ tr^2$之前的概率$ \to 2 - 2\Phi(\dfrac{1}{\sqrt{t}})$.(反直觉)

直觉认为返回后``无记忆'',因此累积时间应当与$ r$同阶.

\subsection{Generating Function Perspectives}
以下的Random Walk中,$ +1,-1$分别具有概率$ p,q$.

\textbf{Waiting Time for a Gain}:设$ \varphi_{n} = \varphi_{n,1}$为初过$ 1$的概率,$ \Phi(x)=\sum_{n=0}^\infty{\varphi_nx^n}$

由定义有$ \varphi_0=0,\varphi_1=p$.

对$ n>1$,必须以$q$的概率走到$ S_1 = -1,\varphi_{v-1}$的概率走到First Return点$ S_v = 0,\varphi_{n-v}$概率走到$ S_n=1$.
于是,$ \varphi_n = q\sum_{k=1}^{n-2}{\varphi_k\varphi_{n-1-k}}$

注意到右边的Convolution形式,可得$ \Phi(x)-px = qx\Phi^2(x)\Rightarrow \Phi(x) = \dfrac{1-\sqrt{1-4pqx^2}}{2qx}$
\\

解法二:$ \Phi(x) = \E{x^n} = p\E{x^n | X_1=1} + q\E{x^n|X_1=-1}.$

$X_1=1\Rightarrow n=1\Rightarrow \E{x^n|X_1=1}=s$

$ X_1=-1\Rightarrow n=1+n_1+n_2$(此处仍按照第一步和初返点来分段)

$ \Rightarrow \E{x^n|X_1=-1}=\E{x^{1+n_1+n_2}}=x\Phi^2(x)$.代入即得$ \Phi(x)$
\\

总会赢的概率:$ \sum{\varphi_n} = \min\{\dfrac{p}{q},1\}$

赢的期望时间:$ \Phi'(1) = \dfrac{1}{p-q}.$ 公平赌博期望无穷次才能获利.
\vspace{0.6cm}

\textbf{Return}:$ u_{2n} = C_{2n}^np^nq^n,U(x) = \sum_{n=0}^\infty u_{2n}x^{2n} = \dfrac{1}{\sqrt{1-4pqx^2}}$

\textbf{First Return}:
设$f_n = f_n^{-}+f_n^+, X_1=-1$时的$ f_{2n}^- = q\varphi_{2n-1}$

$\Rightarrow F^-(x) = \sum_{n=1}^\infty f_{2n}^-x^{2n}=qx\Phi(x)=\dfrac{1-\sqrt{1-4pqx^2}}{2}$

将$ p,q$互换得到$ F^+(x)=F^-(x) \Rightarrow F(x)=\sum_{n=1}^\infty f_nx^n = 1-\sqrt{1-4pqx^2}$
\\

总会返回的概率:$ F(1) = 1-|p-q|$; 永不返回的概率:$ |p-q|$

$ p=\dfrac{1}{2}$时一定会返回,期望时间为$ F'(1)=\infty$.

\subsection{Ruin Problem}
考虑一次赌博,若总资产为$ a$,对于从$z$出发的赌徒,走到$ a$或$0$(破产)代表赌博结束.也即在$0$和$a$处有absorbing barrier的random walk.

\textbf{破产概率:}

设从$ z$出发在$ 0$和$a$处被吸收的概率分别为$ q_z,p_z$,对于$ q_z$,有如下差分方程:

$ q_z = pq_{z+1}+qq_{z-1}, 1\le z \le a-1; q_0=1,q_a=0$

$ p\ne q$时,通解为$ q_z=A+B(\dfrac{q}{p})^z; p = q = \dfrac{1}{2}$时,通解为$ q_z=A+Bz$.代入边界条件,可得

破产概率$ q_z=\begin{cases} \dfrac{(q/p)^a-(q/p)^z}{(q/p)^a - 1} , q\neq p \\ 1-\dfrac{z}{a}, q=p \end{cases}
\to  \begin{cases} 1, p\le q \\ (q/p)^z, p > q\end{cases} (a\to \infty)$

将$ p,q; a-z,z$交换即得$ p_z$的表达式,可见$ p_z+q_z=1$(概率1会终止).
\\

\textbf{持续时间:}

假设持续时间的期望存在有限,为$ D_z$,则可得差分方程:

$ D_z= p(D_{z+1}+1)+q(D_{z-1}+1) = pD_{z+1}+qD_{z-1}+1, D_0=D_a=0$

$ p\neq q$时,有特解$ D_z = \dfrac{z}{q-p}\Rightarrow $通解$ D_z = \dfrac{z}{q-p}+A+B(\dfrac{q}{p})^z$

$ p=q$时,有特解$ D_z = -z^2\Rightarrow $通解$ D_z=-z^2+A+Bz$,代入边界,得

$ D_z = \begin{cases}\dfrac{z}{q-p}-\dfrac{a}{q-p}\dfrac{1-(q/p)^z}{1-(q/p)^a}, p\neq q\\ z(a-z), p=q\end{cases} \to \begin{cases} \infty, p\ge q \\
  \dfrac{z}{q-p}, p < q \end{cases} (a\to \infty)$
\\

\textbf{生成函数:}

设$ u_{z,n}$表示$n$时刻破产(走到0),$ U_z(t) = \sum_{n=0}^\infty u_{z,n}t^n$.

有$ u_{0,n}=u_{a,n} = u_{z,0}=0 (0<z\le a, n\ge 1), u_{0,0}=1$及$ u_{z,n+1}=pu_{z+1,n}+qu_{z-1,n}$

$ \Leftrightarrow U_z(t) = ptU_{z+1}(t)+qtU_{z-1}(t), U_0(t) = 1, U_{a}(t) = 0$

  由特征根法可解出$ U_z(t) = A(t)\lambda_1^z(t)+B(t)\lambda_2^z(t),\lambda_i(t) = \dfrac{1\pm \sqrt{1-4pqt^2}}{2pt}$

  代入边界,得破产时刻的生成函数$ U_z(t) = (\dfrac{q}{p})^z\dfrac{\lambda_1^{a-z}(t) - \lambda_2^{a-z}(t)}{\lambda_1^a(t)-\lambda_2^a(t)}$

  类似可得获胜时刻的生成函数$ V_z(t) = \dfrac{\lambda_1^z(t)-\lambda_2^z(t)}{\lambda_1^a(t) - \lambda_2^a(t)}$

  $ a\to \infty$时,注意Vieta定理可得$ U_z(t) = \lambda_2^z(t),\lambda$取小根.

  半开随机徘徊的破产对应First Passage. 上式表明初过$ -z$的等待时间是$ z$个独立时间和.

  $ U_z(t)$  的显式展开:$ u_{z,n} =
  \dfrac{2^n}{a}p^{\frac{n-z}{2}}q^{\frac{n+z}{2}}\sum_{k=1}^{a-1}[\cos^{n-1}\dfrac{k\pi}{a}\sin\dfrac{k\pi}{a}\sin\dfrac{kz\pi}{a}]$

\subsection{Higher Dimensions}
\textbf{Return:}在一维/二维Random Walk中,质点将以概率1返回初始位置,而三维中概率约为0.35.

一维时, $ u_{2n}\approx \dfrac{1}{\sqrt{\pi n}} = O(n^{-\frac{1}{2}}), \Oldsum u_{2n}$发散,于是$ \Oldsum f_{2n} = 1$,必定会返回.

二维时, $ u_{2n} = \dfrac{1}{4^{2n}}\sum_{k=0}^n\dfrac{(2n)!}{k!k!(n-k)!(n-k)!} = \dfrac{1}{4^{2n}}C_{2n}^n \sum_{k=0}^n (C_n^k)^2 =
\dfrac{(C_{2n}^n)^2}{4^{2n}} = O(n^{-1}). $

三维时,$ u_{2n} = \dfrac{1}{6^{2n}}\sum_{j+k\le n}\dfrac{(2n)!}{j!j!k!k!(n-j-k)!(n-j-k)!} = \dfrac{1}{2^{2n}}C_{2n}^n\sum_{j+k\le n}[\dfrac{n!}{3^n
  j!k!(n-j-k)!}]^2$

当$j=k=\dfrac{n}{3} $时和式内取到最大项,只分析此项有$ u_{2n} = O(n^{-\frac{3}{2}}), \Oldsum u_{2n}$收敛$ \Rightarrow \Oldsum f_{2n}<1$.

三维空间中可以以概率1到达任意一条与坐标轴平行的直线.(投影成二维Random Walk)
\\

\textbf{期望距离:}在$ d$维空间中,设$ n$时刻到达$ (x_{1,n}\cdots x_{d,n})$,考虑距原点距离平方的增量:

$ \E{D_{n+1}^2 - D_{n}^2} = \E{-2\sum_d{x_{d,n}X_d} + \sum_d{X_d^2}} = \E{\sum_d{X_d^2}} = 1\Rightarrow \E{D_n}=\sqrt{n}$

